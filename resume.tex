% Please build with lualatex
% VS Code -> LaTeX Workshop extension settings:
% {
%     "latex-workshop.latex.recipes": [
%         {
%             "name": "lualatex",
%             "tools": ["lualatex"]
%         }
%     ],
%     "latex-workshop.latex.tools": [
%         {
%             "name": "lualatex",
%             "command": "lualatex",
%             "args": [
%                 "-synctex=1",
%                 "-interaction=nonstopmode",
%                 "-file-line-error",
%                 "-output-directory=%OUTDIR%",
%                 "%DOC%"
%             ],
%             "env": {}
%         }
%     ],
%     "latex-workshop.linting.chktex.enabled": true
% }

\documentclass[11pt]{article}

\usepackage{preamble}

% Title and contacts
\newcommand\HUGE{\fontsize{26}{39}\selectfont}
\fancyhead[C]{
	% Title name
	\textbf{\HUGE{Yuchen Zhang}} \\
	\vspace{0.05in} 
	% Email
	\href{mailto:anthony.yuchen@gmail.com}{anthony.yuchen@gmail.com}  | 
	% Phone
	\href{tel:13238688380}{(323) 868-8380} \\
	% Personal website (disabled for now)
	\href{https://anthonyive.github.io/}{anthonyive.github.io}  | 
	% GitHub profile
	\href{https://www.github.com/anthonyive}{github.com/anthonyive}  | 
	% LinkedIn profile
	\href{https://www.linkedin.com/in/anthonyive}{linkedin.com/in/anthonyive} 
}

\begin{document}

%% WORK EXPERIENCE %%
\myHeading{WORK EXPERIENCE}
\subsection*{CAD Software Developer Engineer{\normalfont, Qualcomm Technologies, Inc.,
            \textit{CA} \hfill
            July 2022-Present}}
\begin{compactitem}
    \item Revitalizing verification tools for VLSI front end design
    \item Developing automation scripts for generating SystemVerilog files
    \item Conceptualizing documentation generation tools for Engineers
\end{compactitem}

\subsection*{Student Worker{\normalfont, USC Institute for Creative
            Technologies,
            \textit{CA} \hfill
            September 2021-April 2022}}
\begin{compactitem}
    \item Conducted data preprocessing, analysis, and visualizations on clinical psychology utterance data
    \item Examined conversation segmentation and separations on 200+ GB of audio files using ELAN
    \item Fine-tuned unsupervised Multinomial HMM models for predicting transcript codes with their confidence scores
    \item Implemented supervised RNN and LSTM models for predicting utterance states
    \item Generated LDA topic modeling models for exploring topics across states
\end{compactitem}
%\vspace{0.1in}

\subsection*{Course Producer{\normalfont, USC Viterbi School of Engineering,
            \textit{CA} \hfill
            August 2021-December 2021}}
\begin{compactitem}
    \item Actively assisted 90+ students in Foundations of Data Management
    (DSCI 551)
    \item Unit-tested students' homeworks using Python
    \item Held office hours to help debug students' programming assignments
    and to reinforce
    course content
    \item Helped the Professor in grading homeworks, labs, and proctoring exams, and
    answering 180+ questions on Piazza
\end{compactitem}
%\vspace{0.1in}

% \subsection*{Research Internship{\normalfont, CarmaCam, \textit{Remote} \hfill May 2021-August 2021}}
% \begin{compactitem}
%     \item Upgraded the existing Machine Learning Scoring website with a new UI using Django and Celery
%     % \item Re-designed the CarmaCam web app using Parallel Agile\textregistered\xspace CodeBot\textregistered
% %    \item Aggregated API calls and integrated the website with Node.JS
% 	\item Migrated the database of the ML scoring website from SQLite to MongoDB
% 	\item Improved the database schema of CodeBot using EA Architect
% \end{compactitem}
%\vspace{0.1in}

%\subsection*{Emergency Data Relief Internship (QA track){\normalfont, BroadStreet.io, \textit{Remote} \hfill September 2020-December 2020}}
%\begin{compactitem}
%    \item Maintained quality assurance track for the daily number of COVID-19 cases
%    \item Systematized \href{https://github.com/Anthonyive/broadstreet-qa-automation.git}{QA automation}
%    process using Google Sheet API that cut the working time by more than
%    50\%
%    % \item Investigated trend breaks, diagnosed input errors, and communicated
%    % with team members
%\end{compactitem}

\vspace{0.1in}

%% ACADEMIC PROJECTS %%
\myHeading{ACADEMIC PROJECTS}
\subsection*{Data Mining on the Yelp Dataset \hfill {\normalfont{January 2022-April 2022}}}
\noindent
Goal: To implement algorithms and build recommendation systems on a large dataset
\begin{compactitem}
	\item 200k+ reviews. Code containing 12 files totaling around 1800 LOC (lines of code) using PySpark and MapReduce
	\item Implemented Locality Sensitive Hashing, and different types of collaborative-filtering recommendation systems: Item-based CF, Model-based CF, and Hybrid recommendation system. 
    \item Fine-tuned Model-based CF with XGBoostRegressor
	\item Community detection based on GraphFrames and based on Girvan-Newman Algorithm on an RDD level
\end{compactitem}

\subsection*{The Influence of Pre- \& Post-processing on Document Summarization \hfill {\normalfont{August
        2021-December 2021}}}
\noindent
% \href{https://github.com/Anthonyive/csci-544-project.git}{\faIcon{github}}  \href{https://www.youtube.com/watch?v=oVIVtOPeWEs}{\faIcon{youtube}} \href{https://arxiv.org/abs/2112.01660}{\textbf{arXiv}} 
Goal: To improve existing long document summarization models' performance\hfill\textit{arXiv}: \href{https://arxiv.org/abs/2112.01660}{https://arxiv.org/abs/2112.01660}
\begin{compactitem}
    \item Implemented extractive-based baseline (e.g. TextRank) and Google's T5 text-to-text transformer model
%    \item Inspired team members to implement GPT-3 and XLNet models
    \item Slight improvement in Rouge-1 score (15\%) and substantially improved Rouge-2 scores (53\%) for certain datasets
%    \item Formulated team's workflow by using GitHub actions and branching
\end{compactitem}
%\vspace{0.1in}

\subsection*{Mapping Uncanny Valley \hfill {\normalfont{September
        2020-September 2022}}}
\noindent
% \href{https://github.com/Anthonyive/Research-Mapping-Uncanny-Valley.git}{\faIcon{github}} 
Goal: To help answer what makes text creepy\hfill \textit{arXiv}: \href{https://arxiv.org/abs/2211.05369}{https://arxiv.org/abs/2211.05369}
\begin{compactitem}
    \item Received Best Project Achievement Award and Best Data Science Open and
    Sharing Practices Award
    \item Received Best Cyberphysical Data Science Team Award for a team of six people
    in DataFest Fall 2020
    \item Conducted DNN/RNN model with accuracy up to 96\% and implemented
    multiple NLP techniques
%    \item Studied 20+ papers and ideas, and collaborated with team weekly
    \item Co-authored ``Mapping the Uncanny Valley in Text'' submission in
    ICWSM '22
\end{compactitem}
%\vspace{0.1in}

\subsection*{Walmart Product Search \hfill {\normalfont{January 2021-May 2021}}}
\noindent
% \href{https://github.com/Anthonyive/DSCI-551-Project.git}{\faIcon{github}} 
Goal: To build a full-stack Walmart product search UI
\begin{compactitem}
    \item Leveraged MongoDB as a backend to store 6000 paginated pages from Walmart
    Affiliate API
    \item Built a Flask app for searching and filtering from over 300,000
    product items with over 30 fields each
%    \item Hosted the UI website on AWS EC2 and Route53 using Nginx with SSL
%    certificate
\end{compactitem}
%\vspace{0.1in}

%\subsection*{Title: Analysis of Cyber Phishing Emails \hfill 
%{\normalfont January 2021-May 2021}}
%\noindent
%\href{https://github.com/Anthonyive/DSCI-550-Assignments.git}{\faIcon{github}} 
%Goal: To analyze fraudulent email dataset on Kaggle.
%\begin{compactitem}
% \item Analyzed social engineering attack techniques using NLP (spaCy, 
%     nltk, pre-trained neural nets, etc)
% \item Implemented GPT-2 and DCGAN frameworks to generate attacker 
%     face images and emails
%\end{compactitem}
%\vspace{0.1in}

%\myHeading{LEADERSHIP \& INVOLVEMENT}
%\subsection*{Publicity Director{\normalfont, International Student Advancement 
%Program, East Los Angeles College \hfill	2016-2017}}
%\begin{compactitem}
%    \item Designed fliers and banners for Food Sales, Lunar Festivals, 
%        New Year Fairs, and other activities
%    \item Arranged and presented with club members to help and decorate 
%        club activities
%\end{compactitem}

\vspace{0.1in}

%% EDUCATION %%
\myHeading{EDUCATION}
\subsection*{University of Southern California{\normalfont, \textit{Los Angeles, CA} \hfill May 2022}}
\noindent
Master of Science, Applied Data Science
% \hfill
% CGPA\@ 3.78/4.0

\subsection*{University of California, Los Angeles{\normalfont, \textit{Los Angeles, CA} \hfill August 2020}}
\noindent
Bachelor of Science, Applied Mathematics with a minor in Statistics
% \hfill
% CGPA\@ 3.86/4.0

\vspace{0.1in}

%% SKILLS %%
\myHeading{SKILLS}
\vspace{0.05in}
\begin{compactdesc}
	\item[Topics] Software Development, Natural Language Processing (NLP), Machine Learning, Data Science, Statistics
    \item[Programming Languages] Strong knowledge: Python | Basic knowledge: Perl, Java, R, C\texttt{++}, Bash, etc.
    \item[Database Management] SQLite, MySQL, MongoDB, Amazon DynamoDB, Firebase, Hadoop HDFS
    \item[Tools] PyTorch, TensorFlow, scikit-learn, GitHub, Pandas, PySpark, Distributed Systems, MapReduce, Algorithms, etc.
\end{compactdesc}
\end{document}
