% Please build with lualatex

\documentclass[11pt]{article}

\usepackage{preamble}

\begin{document}
\myHeading{EDUCATION}
\subsection*{University of Southern California{\normalfont, \textit{Los Angeles, CA} \hfill June 2022}}
\noindent
Master of Science, Applied Data Science
\hfill
CGPA: 3.85/4.0

%\vspace{0.1in}

\subsection*{University of California, Los Angeles{\normalfont, \textit{Los Angeles, CA} \hfill August 2020}}
\noindent
Bachelor of Science, Applied Mathematics with a minor in Statistics
\hfill
CGPA: 3.86/4.0

\vspace{0.1in}

\myHeading{SKILLS}
\begin{compactdesc}
	\item[Topics] NLP, Machine Learning, Data Mining, Data Processing
    \item[Programming Languages] Python, R, C++, Bash, etc.
    \item[Database Management] SQLite, MySQL, MongoDB, Amazon DynamoDB, Firebase, Hadoop HDFS
    \item[Tools] PyTorch, TensorFlow, scikit-learn, GitHub, Pandas, Numpy, \LaTeX,
    Algorithms, etc.
\end{compactdesc}

\vspace{0.1in}

\myHeading{WORK EXPERIENCE}
\subsection*{Student Worker{\normalfont, USC Institute for Creative
            Technologies,
            \textit{CA} \hfill
            September 2021-Present}}
\begin{compactitem}
    \item Conducting research on computer vision and multimodal
    human behavior perception
    \item Fine-tuning Multinomial HMM models for predicting transcript codes
\end{compactitem}
%\vspace{0.1in}

\subsection*{Course Producer{\normalfont, USC Viterbi School of Engineering,
            \textit{CA} \hfill
            August 2021-December 2021}}
\begin{compactitem}
    \item Actively assisted 90+ students in Foundations of Data Management
    (DSCI 551)
    \item Unit tested students' homeworks using Python
    \item Held office hours to help debug students' programming assignments
    and to reinforce
    course content
    \item Helped the Professor in grading homeworks, labs, and the project,
    proctoring exams, and
    answering 180+ questions on Piazza
\end{compactitem}
%\vspace{0.1in}

\subsection*{Research Internship{\normalfont, CarmaCam, \textit{Remote} \hfill May 2021-August 2021}}
\begin{compactitem}
    \item Upgraded the existing Django app with a new UI
    % \item Re-designed the CarmaCam web app using Parallel Agile\textregistered\xspace CodeBot\textregistered
    \item Aggregated API calls and integrated the website with Node.JS
\end{compactitem}
%\vspace{0.1in}

\subsection*{Emergency Data Relief Internship (QA track){\normalfont, BroadStreet, \textit{Remote} \hfill September 2020-December 2020}}
\begin{compactitem}
    \item Maintained quality assurance track for the daily number of COVID-19 cases
    \item Systematized \href{https://github.com/Anthonyive/broadstreet-qa-automation.git}{QA automation}
    process using Google Sheet API that cut the working time by more than
    50\%
    % \item Investigated trend breaks, diagnosed input errors, and communicated
    % with team members
\end{compactitem}

\vspace{0.1in}

\myHeading{ACADEMIC PROJECTS}
\subsection*{Title: The Influence of Pre- \& Post-processing on Document Summarization \hfill {\normalfont August
        2021-December 2021}}
\noindent
\href{https://github.com/Anthonyive/csci-544-project.git}{\faIcon{github}}  \href{https://www.youtube.com/watch?v=oVIVtOPeWEs}{\faIcon{youtube}} \href{https://arxiv.org/abs/2112.01660}{\textbf{arXiv}} Goal: To improve existing long document summarization model's performance.
\begin{compactitem}
    \item Implemented extractive-based baseline (e.g. TextRank) and Google's T5 text-to-text transformer model
    \item Inspired team members to implement GPT-3 and XLNet models
    \item Slight improved in R-1 score (15\%) and substantially improved R-2 scores (53\%) for certain datasets
    \item Formulated team's workflow by using GitHub actions and branching
\end{compactitem}
%\vspace{0.1in}

\subsection*{Title: Mapping Uncanny Valley \hfill {\normalfont September
        2020-May 2021}}
\noindent
\href{https://github.com/Anthonyive/Research-Mapping-Uncanny-Valley.git}{\faIcon{github}} Goal: To help answer what makes text creepy.
\begin{compactitem}
    \item Best Project Achievement Award and Best Data Science Open and
    Sharing Practices Award
    \item Best Cyberphysical Data Science Team Award for a team of six people
    in DataFest Fall 2020
    \item Conducted DNN/RNN model with accuracy up to 96\% and implemented
    multiple NLP techniques
    \item Communicated and collaborated with professor and team members weekly
    \item Co-authored ``Mapping the Uncanny Valley in Text'' submission in
    ICWSM 22
\end{compactitem}
%\vspace{0.1in}

\subsection*{Title: Walmart Product Search \hfill {\normalfont January 2021-May 2021}}
\noindent
\href{https://github.com/Anthonyive/DSCI-551-Project.git}{\faIcon{github}} Goal: To build a full-stack Walmart product search UI.
\begin{compactitem}
    \item Leveraged MongoDB as a backend to store 6000 paginated pages from Walmart
    Affiliate API
    \item Built a Flask app for searching and filtering from over 300,000
    product items with over 30 fields each
%    \item Hosted the UI website on AWS EC2 and Route53 using Nginx with SSL
%    certificate
\end{compactitem}
%\vspace{0.1in}

%\subsection*{Title: Analysis of Cyber Phishing Emails \hfill 
%{\normalfont January 2021-May 2021}}
%\noindent
%\href{https://github.com/Anthonyive/DSCI-550-Assignments.git}{\faIcon{github}} 
%Goal: To analyze fraudulent email dataset on Kaggle.
%\begin{compactitem}
% \item Analyzed social engineering attack techniques using NLP (spaCy, 
%     nltk, pre-trained neural nets, etc)
% \item Implemented GPT-2 and DCGAN frameworks to generate attacker 
%     face images and emails
%\end{compactitem}
%\vspace{0.1in}

%\myHeading{LEADERSHIP \& INVOLVEMENT}
%\subsection*{Publicity Director{\normalfont, International Student Advancement 
%Program, East Los Angeles College \hfill	2016-2017}}
%\begin{compactitem}
%    \item Designed fliers and banners for Food Sales, Lunar Festivals, 
%        New Year Fairs, and other activities
%    \item Arranged and presented with club members to help and decorate 
%        club activities
%\end{compactitem}

\end{document}
